
\documentclass[10pt,twocolumn]{article} 
\usepackage{simpleConference}
%\documentclass{article}
%\usepackage{arxiv}
%\usepackage[margin=1in]{geometry}



\usepackage[utf8]{inputenc} % allow utf-8 input
\usepackage{hyperref}       % hyperlinks
\usepackage{url}            % simple URL typesetting
\usepackage{booktabs}       % professional-quality tables
\usepackage{amsfonts}       % blackboard math symbols
\usepackage{amsmath}
\usepackage{amssymb}
\usepackage{nicefrac}       % compact symbols for 1/2, etc.
\usepackage{graphicx}
\usepackage{natbib}
\usepackage{doi}
\usepackage{wrapfig}


\usepackage{listings}
\usepackage{xcolor}

\definecolor{codegreen}{rgb}{0,0.6,0}
\definecolor{codegray}{rgb}{0.5,0.5,0.5}
\definecolor{codepurple}{rgb}{0.58,0,0.82}
\definecolor{backcolour}{rgb}{0.97,0.97,0.97}

\lstdefinestyle{mystyle}{
	backgroundcolor=\color{backcolour},   
	commentstyle=\color{codegreen},
	keywordstyle=\color{magenta},
	numberstyle=\tiny\color{codegray},
	stringstyle=\color{codepurple},
	basicstyle=\ttfamily\footnotesize,
	breakatwhitespace=false,         
	breaklines=true,                 
	captionpos=b,                    
	keepspaces=true,                 
	numbers=left,                    
	numbersep=5pt,                  
	showspaces=false,                
	showstringspaces=false,
	showtabs=false,                  
	tabsize=2
}

\lstset{style=mystyle}


\title{Going beyond the Final Linear Layer: Enhancing Decision Boundaries}

\author{Michael Majurski, David Chapman\thanks{Information Technology Lab, National Institute of Standards and Technology, michael.majurski@nist.gov}
}




%%% Add PDF metadata to help others organize their library
%%% Once the PDF is generated, you can check the metadata with
%%% $ pdfinfo template.pdf
\hypersetup{
	pdftitle={Beyond the Final Linear Layer},
	pdfauthor={Michael Majurski, David Chapman},
	pdfkeywords={AI,Semi-Supervised,Classification,FixMatch},
}


\begin{document}
\maketitle

\begin{abstract}
Semi-Supervised image classification has become a playground for exploring new ideas in extracting signal from unannotated image datasets.
This paper presents a novel extension to any image classification architecture which improves accuracy in low-label regimes. 
We extend the FixMatch \cite{sohn2020fixmatch} training scheme with our novel last layers and demonstrate test accuracy improvement. 
The novelty consists of 2 elements, first we replace the last linear layer with a GMM trained via backprop, and impose class-wise constraints on the embedding space the GMM operates on.
These methods match published SOTA 250 label Cifar10 \cite{cifar10} results and come close to matching SOTA in the 40 label regime without the significant model complexity of methods like SimMatchV2 \cite{zheng2023simmatchv2}.
Our method achieves 94.8\% and 94.2\% accuracy with 250 and 40 Cifar10 labels respectively.
\end{abstract}


\section{Introduction}
Semi-Supervised learning attempts to leverage the abundance of unlabeled image data to improve deep learning based model performance under limited training data regimes \cite{zhu2022introduction,li2019safe,hady2013semi}.
The early sucesses of deep learning based methods relied on large populations of annotated 


%PyTorch \cite{pytorch} 



\section{Related Work}



\subsection{Pseudo-Labeling}

\subsection{Consistency Regularization}

\subsection{Contrastive Learning}

%
%\begin{lstlisting}[language=Python]
%class Factor:
%	"""Class to store DEX (Design of Experiment) factor data.
%	"""
%	
%	def __init__(self, 
%				levels: list[any], 
%				jitter: float = None, 
%				rso: np.random.RandomState = None, 
%				requested_level: any = None):
%		"""
%		Initialize a Factor class instance with the provided levels. This includes sampling from those levels to pick an instance, apply any jitter, constructing the final dex factor value.
%		
%		Args:
%		levels: The levels to select from for this factor.
%		jitter: Any jitter to apply to the selected level before saving it into the value. I.e. value = 5.0 +- 1.2, results in a value between 3.8 and 6.2
%		rso: The ransom state object used to do the sampling.
%		requested_level: The requested level for this factor. This must a value within the provided levels list.
%		"""
%\end{lstlisting}


\section{Methodology}

\subsection{KMeans}

\subsection{Axis Aligned Differentiable Gaussian Mixture Model}


\subsection{Embedding Constraints}



\section{Experiments}


\subsection{Ablation Study}



\section{Conclusions}





\bibliographystyle{unsrt}
\bibliography{refs}
	
	
\end{document}